\chapter{Diagnostics}
\label{chap:Diagnostics}

\section{Output files of GKV}
\label{sec:Output files of GKV}

When finishing a run of GKV, all simulation output will be dumped in the output directory \texttt{DIR} set in the \texttt{run/shoot} script (for example in Table \ref{table:run/shoot}, \texttt{DIR=/data/lng/maeyama/gkv\_training/test01/}), as classified into the following directories,
\begin{itemize}
  \item \texttt{DIR/}
  \begin{itemize}
    \item \texttt{log/} (Log files on simulation runs)
    \item \texttt{cnt/} (Binary data for restart)
    \item \texttt{fxv/} (Binary data of distribution functions $\tilde{f}_{\mathrm{s}\bm{k}}(k_x,k_y,v_\parallel,\mu)$ at several positions of $z$)
    \item \texttt{phi/} (Binary data of potentials, fluid moments, and variables in the entropy balance equation)%(Binary data of potentials $\tilde{\phi}_{\bm{k}}$, $\tilde{A}_{\parallel\bm{k}}$, fluid moments $\tilde{n}_{\mathrm{s}\bm{k}}$, $\tilde{u}_{\parallel\mathrm{s}\bm{k}}$, $\tilde{p}_{\parallel\mathrm{s}\bm{k}}$, $\tilde{p}_{\perp\mathrm{s}\bm{k}}$, $\tilde{q}_{\parallel\parallel\mathrm{s}\bm{k}}$, $\tilde{q}_{\parallel\perp\mathrm{s}\bm{k}}$, and variables in the entropy balance equation)
    \item \texttt{hst/} (Ascii data of the GKV standard output)
    \item ... (Others are back up of the source code and environmental settings)
  \end{itemize}
\end{itemize}
List of GKV output is summarized in Appendix \ref{sec:List of GKV output}.

Users may diagnose these output data by themselves. The present version of GKV provides two post-processing tools, \texttt{fig\_stdout} and \texttt{diag}, which are contained in \texttt{gkvp\_f0.48/extra\_tools/}.




\section{PDF generating script for ASCII output: fig\_stdout}
\label{sec:PDF generating script for ASCII output: fig_stdout}

Noting that \texttt{fig\_stdout} requires \texttt{gnuplot} later than version 5.0.\\ 
Expanding \texttt{gkvp\_f0.48/extra\_tools/fig\_stdout.tar.gz} under the output directory \texttt{DIR},
\begin{itemize}
  \item \texttt{DIR/}
  \begin{itemize}
    \item \texttt{fig\_stdout/}
    \begin{itemize}
      \item \texttt{make\_pdf.csh} (Script for making a PDF sheet)
      \item \texttt{src/} (Script for gnuplot)
      \item \texttt{pdf/} (Directory to be store the created PDF sheet)
      \item \texttt{eps/} (Directory to be store the plotted EPS files)
      \item \texttt{data/} (Directory to be store raw data of GKV standard output)
    \end{itemize}
  \end{itemize}
\end{itemize}
and typing the following commands,\\
~~~~\texttt{cd fig\_stdout/}\\
~~~~\texttt{./make\_pdf.csh  clean}\\
~~~~\texttt{./make\_pdf.csh}\\
users obtain a PDF sheet of the GKV standard output in \texttt{fig\_stdout/pdf/}.



\section{Post-processing program for BINARY output: diag}
\label{sec:Post-processing program for BINARY output: diag}

\subsection{What is diag?}
One difficulty of users may read GKV binary output which are decomposed by MPI. Post-processing program \texttt{diag} helps to read GKV binary output of a desired quantity at a desired time step. Since the read quantity is constructed as a global variable, e.g., $\tilde{\phi}_{\bm{k}}(\texttt{-nx:nx,0:global\_ny,-global\_nz:global\_nz-1})$ (not a local variable decomposed by MPI $\tilde{\phi}_{\bm{k}}(\texttt{-nx:nx,0:ny,-nz:nz-1})$), users do not need to be conscious of MPI parallelization of GKV. The main program \texttt{diag\_main.f90} calls each diagnostics module \texttt{out\_******.f90}, which should be encapsulated so as to avoid interference and misuse. Reading GKV binary file is done by calling \texttt{diag\_rb} module in each diagnostics module. Although there are some diagnostics modules implemented, users can design a new diagnostics module by themselves.

\subsection{How to use diag}
Expanding \texttt{gkvp\_f0.48/extra\_tools/v29diag.tar.gz}, one finds source codes of \texttt{diag}.
\begin{itemize}
  \item \texttt{v29diag/}
  \begin{itemize}
    \item \texttt{Makefile}
    \item \texttt{go.diag} (Batch script)
    \item \texttt{backup/}
    \item \texttt{plotfile/} (Sample file for gnuplot)
    \item \texttt{src}
    \begin{itemize}
      \item \texttt{diag\_header.f90} (Module for setting grid resolutions and MPI processes in GKV)
      \item \texttt{diag\_main.f90} (Main program calling each diagnostics module)
      \item \texttt{diag\_rb.f90} (Module for reading GKV binary output)
      \item \texttt{diag\_******.f90} (Module for other settings)
      \item ...
      \item \texttt{out\_******.f90} (Module for each diagnostics)
      \item ...
    \end{itemize}
  \end{itemize} 
\end{itemize}
How to use \texttt{diag} is in the following steps:
\begin{enumerate}
  \item Setting parameters in \texttt{v29diag/src/diag\_header.f90}
\begin{table}[h!]
  \caption{\texttt{v29diag/src/diag\_header.f90}}
  \centering
  \begin{tabular}{| l |}
  \hline
  ~~~~$\vdots$ \\
!\%\%\% DIAG parameters \%\%\%\\
  ~~integer, parameter :: snum = 1      ! begining of simulation runs\\
  ~~integer, parameter :: enum = 1      ! end of simulation runs\\
!\%\%\%\%\%\%\%\%\%\%\%\%\%\%\%\% ~~ ! \textit{Set run numbers covering diagnosed time range.}\\
\\
!\%\%\% GKV parameters \%\%\%\\
  ~~integer, parameter :: nxw = 2, nyw = 8\\
  ~~integer, parameter :: nx = 0, global\_ny = 5 ! 2/3 de-aliasing rule\\
  ~~integer, parameter :: global\_nz = 64, global\_nv = 24, global\_nm = 15\\
  ~~integer, parameter :: nzb = 3, \& ! the number of ghost grids in z\\
  ~~~~~~~~~~~~~~~~~~~~~~~~~~~~~~nvb = 3 ! the number of ghost grids in v and m\\
  ~~integer, parameter :: nprocw = 1, nprocz = 4, nprocv = 2, nprocm = 2, nprocs = 2\\
!\%\%\%\%\%\%\%\%\%\%\%\%\%\%\%\% ~~ ! \textit{These should be same as} \texttt{gkvp\_f0.48\_header.f90}.\\
  ~~~~$\vdots$ \\
  \hline
  \end{tabular}
\end{table}
  \item Calling diagnostics modules in \texttt{v29diag/src/diag\_main.f90}
  \item Setting the output directory of GKV, \texttt{DIR}, in \texttt{go.diag}
  \item Compile \& Execution
  \item Output data is dumped in \texttt{\$DIR/post/}. 
\end{enumerate}

\subsection{Examples of diag}
\begin{table}[h!]
  \caption{Example of \texttt{v29diag/src/diag\_main.f90}, to output 2D electrostatic potential in $x$-$y$ plane at a given $z$, $\tilde{\phi}(x,y)$}
  \centering
  \begin{tabular}{| l l |}
  \hline
  PROGRAM diag&\\
  ~~~~\vdots&\\
  ~~use out\_mominxy, only : phiinxy  & ! \textit{Use corresponding diagnostics module}\\
  ~~implicit none&\\
  ~~integer :: giz, loop&\\
  ~~~~\vdots&\\
  ~~giz = 0 & ! \textit{Set diagnosed grid in $z$} ($-\texttt{global\_nz} \leq \texttt{giz} \leq \texttt{global\_nz}-1$)\\
  ~~loop = 100 & ! \textit{Set diagnosed time step} ($\texttt{time} = \texttt{dtout\_ptn} * \texttt{loop}$)\\
  ~~call phiinxy( giz, loop ) & ! \textit{Output} $\tilde{\phi}(x,y)$ \textit{at} $\texttt{giz}=0$, $\texttt{loop}=100$\\
  ~~~~\vdots&\\
  END PROGRAM diag&\\
  \hline
  \end{tabular}
\end{table}
\begin{table}[h!]
  \caption{Example of \texttt{v29diag/src/diag\_main.f90}, to output 1D electrostatic potential in the field-aligned $z$ coordinate of a given mode $k_x, k_y$, $\tilde{\phi}_{\bm{k}}(z)$}
  \centering
  \begin{tabular}{| l l |}
  \hline
  PROGRAM diag&\\
  ~~~~\vdots&\\
  ~~use out\_mominz, only : phiinz  & ! \textit{Use corresponding diagnostics module}\\
  ~~implicit none&\\
  ~~integer :: mx, gmy, loop&\\
  ~~~~\vdots&\\
  ~~mx = 0 & ! \textit{Set diagnosed radial mode number $k_x$} ($-\texttt{nx} \leq \texttt{mx} \leq \texttt{nx}-1$)\\
  ~~gmy = 6 & ! \textit{Set diagnosed bi-normal mode number $k_y$} ($0 \leq \texttt{gmy} \leq \texttt{global\_ny}$)\\
  ~~loop = 100 & ! \textit{Set diagnosed time step} ($\texttt{time} = \texttt{dtout\_ptn} * \texttt{loop}$)\\
  ~~call phiinz( mx, gmy, loop ) & ! \textit{Output} $\tilde{\phi}_{\bm{k}}(z)$ \textit{at} $\texttt{mx}=0$, $\texttt{gmy}=6$, $\texttt{loop}=100$\\
  ~~~~\vdots&\\
  END PROGRAM diag&\\
  \hline
  \end{tabular}
\end{table}

For more details, Appendix \ref{sec:Data-reading module diag_rb in the post-processing program diag} explains how \texttt{diag\_rb} module read GKV binary data, and Appendix \ref{sec:Diagnostics modules in the post-processing program diag} shows some examples of diagnostics modules.




%\begin{thebibliography}{99}
%\bibitem{Warbet2010NF}
%  W
%\bibitem{Horton1999RMP}
%  H
%\bibitem{Garbet2010NF}
%  G
%\end{thebibliography}