\chapter{Formulation}
\label{chap:Formulation}

* Blue-colored sentences are physical assumptions used in GKV \cite{Watanabe2006NF}. 
\textcolor{red}{This manual is based on the GKV version gkvp\_f0.48.}

\section{Governing equations}
\label{sec:Governing equations}

\textcolor{blue}{One derives gyrokinetic equations based on the following gyrokinetic ordering \cite{Garbet2010NF},
\begin{align}
  \frac{\tilde{f}}{F} \sim \frac{e\tilde{\phi}}{T} \sim \frac{\tilde{B}}{B} \sim \frac{k_\parallel}{k_\perp} \sim \frac{\omega}{\Omega} \equiv \delta \ll 1. \label{eq:gyrokinetic_ordering}
\end{align}}

GKV follows $\delta f$ gyrokinetics, where distribution functions are split into equilibrium and perturbed parts $\mathcal{F}=F+\tilde{f}$. Additionally, there are some subsidiary assumptions:
\textcolor{blue}{\begin{itemize}
\item separation of the equilibrium and perturbed scale lengths $|\nabla F|/F \ll |\nabla \tilde{f}|/f$ || decouples neoclassical physics from turbulent dynamics and treats flute-type perturbations
\item low $\beta$ value || justifies neglect of compressional magnetosonic waves $\tilde{B}_\parallel$ and higher-order correction in $\beta$, but retains shear Alfv\'{e}nic dynamics $\tilde{A}_\parallel$
\item low equilibrium flows $v_\mathrm{eq.} \ll v_\mathrm{th}$ || the present version of GKV cannot treat equilibrium flows
\item the equilibrium distribution function is to be a local Maxwellian $F=F_\mathrm{M}=n \left(\frac{m}{2 \pi T}\right)^\frac{3}{2} e^{-\frac{mv_\parallel^2}{2T}-\frac{\mu B}{T}}$
\item the equilibrium magnetic field satisfies the MHD equilibrium $\nabla P = \bm{J} \times \bm{B}$
\end{itemize}}

Then, the $\delta f$ gyrokinetic Vlasov-Poisson-Amp\`{e}re equations are
\begin{align}
  &\frac{\partial \tilde{f}_\mathrm{s}}{\partial t} + \left( v_\parallel \frac{\bm{B} + \tilde{\bm{B}}_\perp}{B} + \tilde{\bm{v}}_\mathrm{E} + \bm{v}_\mathrm{sG} + \bm{v}_\mathrm{sC} \right) \cdot \nabla \left( \tilde{f}_\mathrm{s} + \frac{e_\mathrm{s} F_\mathrm{sM}}{T_\mathrm{s}} J_{0\mathrm{s}} \tilde{\phi} \right) \nonumber \\
  &- \frac{\mu \nabla_\parallel B}{m_\mathrm{s}} \frac{\partial}{\partial v_\parallel} \left( \tilde{f}_\mathrm{s} + \frac{e_\mathrm{s} F_\mathrm{sM}}{T_\mathrm{s}} J_{0\mathrm{s}} \tilde{\phi} \right) + \frac{e_\mathrm{s} F_\mathrm{sM}}{T_\mathrm{s}} \left[ v_\parallel \frac{\partial J_{0\mathrm{s}} \tilde{A}_\parallel}{\partial t} - \bm{v}_{\mathrm{s}*} \cdot \nabla J_{0\mathrm{s}} (\tilde{\phi} - v_\parallel \tilde{A}_\parallel) \right] = C_\mathrm{s}, 
  \label{eq:vlasovinreal}\\
  &\left[ \nabla_\perp^2 - \frac{1}{\varepsilon_0} \sum_\mathrm{s} \frac{e_\mathrm{s}^2 n_\mathrm{s}}{T_\mathrm{s}} \left( 1 - \Gamma_{0\mathrm{s}} \right) \right] \tilde{\phi} = - \frac{1}{\varepsilon_0} \sum_\mathrm{s} e_\mathrm{s} \int dv^3 J_{0\mathrm{s}} \tilde{f}_\mathrm{s},
  \label{eq:poissoninreal} \\
  &\nabla_\perp^2 \tilde{A}_\parallel = - \mu_0 \sum_\mathrm{s} e_\mathrm{s} \int dv^3 J_{0\mathrm{s}} v_\parallel \tilde{f}_\mathrm{s},
  \label{eq:ampereinreal}
\end{align}
where the gyrophase-average operators $J_{0\mathrm{s}} = \oint (d\xi/2\pi) e^{\bm{\rho}_\mathrm{s} \cdot \nabla} = \oint (d\xi/2\pi) e^{-\bm{\rho}_\mathrm{s} \cdot \nabla}$ and $\Gamma_{0\mathrm{s}} = \int dv^3 (F_\mathrm{sM}/n_\mathrm{s}) J^2_{0\mathrm{s}}$ are used with the gyroradius vector $\bm{\rho}_\mathrm{s} = \bm{b} \times m_\mathrm{s} \bm{v} / (e_\mathrm{s}B)$. The electric and magnetic fields are $\tilde{\bm{E}} = - \nabla (J_{0\mathrm{s}} \tilde{\phi}) - \bm{b} \partial \tilde{A}_\parallel /\partial t$ and $\tilde{\bm{B}}_\perp = \nabla (J_{0\mathrm{s}} \tilde{A}_\parallel) \times \bm{b}$. The $\bm{E} \times \bm{B}$, grad-B, curvature, diamagnetic drift velocities are respectively given by $\tilde{\bm{v}}_\mathrm{E} = \bm{b} \times \nabla (J_{0\mathrm{s}} \tilde{\phi})/B$, $\bm{v}_\mathrm{sG} = \bm{b} \times \mu \nabla B/(e_\mathrm{s}B)$, $\bm{v}_\mathrm{sC} = \bm{b} \times m_\mathrm{s} v_\parallel^2 \bm{b} \cdot \nabla \bm{b}/(e_\mathrm{s}B)$ and $\bm{v}_{\mathrm{s}*} = \bm{b} \times [T_\mathrm{s} \nabla \ln n_\mathrm{s} + ( m_\mathrm{s} v_\parallel^2 /2+ \mu B - 3 T_\mathrm{s}/2) \nabla \ln T_\mathrm{s} ] /(e_\mathrm{s}B)$. $C_s$ is the linearized collision term on the species $s$ and will be explained in Section \ref{sec:Collision operator}. The nonlinear term in the Vlasov eq.  (denoted $\mathcal{N}_\mathrm{s}$ below), which originates from $\bm{E} \times \bm{B}$ and $v_\parallel \tilde{\bm{B}}_\perp/B$ advections of $\tilde{f}$ and $\tilde{\bm{E}} \cdot \tilde{\bm{B}}_\perp$ acceleration of $F$, can be rewritten as,
\begin{align}
  \mathcal{N}_\mathrm{s} &= \left( \tilde{\bm{v}}_\mathrm{E} + v_\parallel \frac{\tilde{\bm{B}}_\perp}{B} \right) \cdot \nabla \tilde{f}_\mathrm{s} + \frac{e_\mathrm{s} \tilde{\bm{E}}}{m_\mathrm{s}} \cdot \frac{\tilde{\bm{B}}_\perp}{B} \left(- \frac{m_\mathrm{s} v_\parallel}{T_\mathrm{s}} F_\mathrm{Ms} \right) \nonumber \\
  &= \frac{\bm{b}}{B} \cdot \nabla \left( J_{0\mathrm{s}} \tilde{\phi} - v_\parallel J_{0\mathrm{s}} \tilde{A}_\parallel \right) \times \nabla \left( \tilde{f}_\mathrm{s} + \frac{e_\mathrm{s} F_\mathrm{Ms}}{T_\mathrm{s}} J_{0\mathrm{s}} \tilde{\phi} \right), 
\end{align}
respectively.



\section{Geometry and coordinates} 
\label{sec:Geometry and coordinates}

When an equilibrium magnetic field is known, one can construct a flux coordinate $(\rho_f, \theta_f, \varphi_f)$ such that,
\begin{align}
  \bm{B} = \nabla \Psi_p(\rho_f) \times \nabla [q(\rho_f)\theta_f - \varphi_f],
\end{align}
where we use the safety factor $q(\rho_f) = d\Psi_t/d\Psi_p$ and the toroidal and poloidal flux $\Psi_p(\rho_f)$ and $\Psi_t(\rho_f)$. GKV employs Clebsch-type coordinate as
\begin{align}
  x &= c_x (\rho_f - \rho_{f0}), \\
  y &= c_y [q(\rho_f) \theta_f - \varphi_f], \\
  z &= \theta_f,
\end{align}
where $\rho_{f0}$, $c_x$ and $c_y$ are constant. We refer $(x, y, z)$ as the radial, field-line-label, and field-aligned coordinates, respectively. Using this GKV coordinates, the equilibrium magnetic field is represented by 
\begin{align}
  \bm{B} = c_b \nabla x \times \nabla y = \frac{c_b}{\sqrt{g}} \frac{\partial \bm{r}}{\partial z},
\end{align}
where $c_b = (d\Psi_p/d\rho_f)/(c_x c_y)$ and $\sqrt{g} = (\nabla x \cdot \nabla y \times \nabla z)^{-1}$. 

Simulation domain of GKV is based on the local flux-tube model \cite{Beer1995PoP}. Using flute approximation for perturbed quantities $k_\perp \gg k_\parallel$ (consistent with the gyrokinetic ordering Eq. (\ref{eq:gyrokinetic_ordering})), vector differential operators in gyrokinetic Eqs. (\ref{eq:vlasovinreal})-(\ref{eq:ampereinreal}) become
\begin{align}
  \nabla_\parallel \tilde f &= \bm{b} \cdot \nabla \tilde f = \frac{c_b}{B \sqrt{g}} \frac{\partial \tilde f}{\partial z}, \\
  \nabla^2 \tilde f &= \frac{1}{\sqrt{g}} \frac{\partial}{\partial r^i} \left[ \sqrt{g} \left( \frac{\partial \tilde f}{\partial r^j} \nabla r^j \right) \cdot \nabla r^i \right] \nonumber \\
  &\simeq g^{xx} \frac{\partial^2 \tilde f}{\partial x^2} + 2 g^{xy} \frac{\partial^2 \tilde f}{\partial x \partial y} + g^{yy} \frac{\partial^2 \tilde f}{\partial y^2}, \\
  \bm{b} \times \nabla \tilde h \cdot \nabla \tilde f &= \bm{b} \cdot \left(\frac{\partial \tilde h}{\partial r^i} \nabla r^i \times \frac{\partial \tilde f}{\partial r^j} \nabla r^j \right) \nonumber \\
  &\simeq \frac{B}{c_b} \left( \frac{\partial \tilde h}{\partial x}\frac{\partial \tilde f}{\partial y} - \frac{\partial \tilde h}{\partial y}\frac{\partial \tilde f}{\partial x} \right), \\
  \bm{b} \times \nabla H \cdot \nabla \tilde f &\simeq \frac{B}{c_b} \left( \frac{\partial H}{\partial x}\frac{\partial \tilde f}{\partial y} - \frac{\partial H}{\partial y}\frac{\partial \tilde f}{\partial x} \right) \nonumber \\
  &+ \frac{\partial H}{\partial z} \left( \frac{g^{xz}g^{yx} - g^{xx}g^{yz}}{B/c_b} \frac{\partial \tilde f}{\partial x} + \frac{g^{xz}g^{yy} - g^{xy}g^{yz}}{B/c_b} \frac{\partial \tilde f}{\partial y} \right),
\end{align}
where $g^{ij}=\nabla r^i \cdot \nabla r^j$ denotes the metric tensor.

\textcolor{blue}{Since the magnetic curvature can be replaced by
\begin{align}
  \bm{b}\cdot\nabla\bm{b} = \frac{\nabla_\perp B}{B} + \frac{\nabla P}{B^2/\mu_0},
\end{align}
when the equilibrium satisfies the MHD equilibrium, $\nabla P = \bm{J} \times \bm{B}$ and $\nabla \times \bm{B} = \mu_0 \bm{J}$,} the magnetic (i.e., grad-B and curvature) drift velocity is given by
\begin{align}
  \bm{v}_\mathrm{sG} + \bm{v}_\mathrm{sC} = \frac{1}{e_\mathrm{s} B} \bm{b} \times \left( \frac{m_\mathrm{s} v_\parallel^2 + \mu B}{B} \nabla B + \frac{m_\mathrm{s} v_\parallel^2}{B^2/ \mu_0} \nabla P \right),
\end{align}
and then the magnetic and diamagnetic drift terms are
\begin{align}
  (\bm{v}_\mathrm{sG} + \bm{v}_\mathrm{sC}) \cdot \nabla (J_0 \tilde{\phi}) &= \frac{m_\mathrm{s} v_\parallel^2 + \mu B}{e_\mathrm{s} c_b} \left( K_x \frac{\partial J_0 \tilde{\phi}}{\partial x} + K_y \frac{\partial J_0 \tilde{\phi}}{\partial y} \right) + \frac{m_\mathrm{s} v_\parallel^2}{e_\mathrm{s} c_b} \frac{dP/dx}{B^2/\mu_0} \frac{\partial J_0 \tilde{\phi}}{\partial y}, \\
  \bm{v}_\mathrm{s*} \cdot \nabla (J_0 \tilde{\phi}) &= - \frac{T_\mathrm{s}}{e_\mathrm{s} c_b} \left[ \frac{1}{L_{n\mathrm{s}}} + \left( \frac{m_\mathrm{s} v_\parallel^2}{2T_\mathrm{s}} + \frac{\mu B}{T_\mathrm{s}} - \frac{3}{2} \right) \frac{1}{L_{T\mathrm{s}}} \right] \frac{\partial J_0 \tilde{\phi}}{\partial y},
\end{align}
where
\begin{align}
  K_x &= - \frac{\partial \ln B}{\partial y} + \frac{g^{xz} g^{yx} - g^{xx} g^{yz}}{B^2/c_b^2} \frac{\partial \ln B}{\partial z}, 
  \label{eq:kkx} \\
  K_y &= \frac{\partial \ln B}{\partial x} + \frac{g^{xz} g^{yy} - g^{xy} g^{yz}}{B^2/c_b^2} \frac{\partial \ln B}{\partial z},
  \label{eq:kky}
\end{align}
and the density and temperature scale lengths $L_{n\mathrm{s}} = - (d\ln n_\mathrm{s}/dx)^{-1}$, $L_{T\mathrm{s}} = - (d\ln T_\mathrm{s}/dx)^{-1}$, and total pressure gradient $dP/dx = d (\sum_\mathrm{s} n_\mathrm{s} T_\mathrm{s})/dx = - \sum_\mathrm{s} n_\mathrm{s} T_\mathrm{s} (L_{n\mathrm{s}}^{-1}+L_{T\mathrm{s}}^{-1})$.




\section{Local approximation}
\label{sec:Local approximation}

Simulation box $-L_x \leq x < L_x$, $-L_y \leq y < L_y$, $-N_\theta \pi < z < N_\theta \pi$ gives flux-tube domain aligned to the equilibrium magnetic field. 

\textcolor{blue}{By assuming the perpendicular scale separation of equilibrium and perturbed quantities, the equilibrium quantities can be evaluated by the value at the center of flux-tube domain $x=0$ or equivalently $\rho_f = \rho_{f0}$. When one considers an axisymmetric equilibrium $\partial_y=0$, the equilibrium quantities are independent to $x$ and $y$, i.e., $F = F(z,v_\parallel,\mu)$, $B = B(z)$, and so on. In a non-axisymmetric equilibrium case, one may treat a thin flux-tube domain not only in $x$ but also in $y$ direction and evaluate the equilibrium quantities at $x=0$ and $y=0$.}





\section{Pseudo-periodic boundary condition along a field line}
\label{sec:Pseudo-periodic boundary condition along a field line}
Since the equilibrium quantities are independent to perpendicular $x$ and $y$ directions, one expand the distribution function and electromagnetic potentials by means of Fourier basis,
\begin{align}
  \tilde{f}_\mathrm{s}(\bm{x},v_\parallel,\mu,t) &= \sum_{k_x}\sum_{k_y} \tilde{f}_{\mathrm{s}\bm{k}} (z,v_\parallel,\mu,t) e^{i(k_xx+ik_yy)}\\
  \tilde{\phi}_(\bm{x}',t) &= \sum_{k_x}\sum_{k_y} \tilde{\phi}_{\bm{k}} (z,t) e^{i(k_xx'+ik_yy')}\\
  J_{0\mathrm{s}} \tilde{\phi}(\bm{x},\mu,t) &= \sum_{k_x}\sum_{k_y} J_0(k_\perp \rho_{\mathrm{ts}}) \tilde{\phi}_{\bm{k}} (z,t) e^{i(k_xx+ik_yy)}\end{align}
where $\bm{x}$ is the gyrocenter coordinates and $\bm{x}'=\bm{x}+\bm{\rho}_\mathrm{s}$ is the particle-position coordinates.

Additionally, considering the torus periodicity constraint $\tilde{\phi}(\rho_f,\theta_f+2N_\theta\pi,\varphi_f) = \tilde{\phi}(\rho_f,\theta_f,\varphi_f)$, one finds the pseudo-periodic boundary condition along a field line,
\begin{align}
  &\tilde{\phi}_{k_x+\delta k_x,k_y} (z+2N_\theta\pi) C_{k_y} = \tilde{\phi}_{k_x,k_y}(z),
  \label{eq:boundarycondition}
\end{align}
where $\delta k_x = -2N_\theta \pi \hat{s} k_y, C_{k_y} = \exp (i2N_\theta \pi k_y c_y q_0)$. This conversion along a field line physically means twisting of the mode by the parallel streaming in the presence of magnetic shear.





\section{Collision operator}
\label{sec:Collision operator}

The present version of GKV equips three types of gyrokinetic model collision operators, operating on the non-adiabatic part of the distribution function $\tilde{g}_{\mathrm{s}\bm{k}} = \tilde{f}_{\mathrm{s}\bm{k}} + \frac{e_\mathrm{s} F_\mathrm{Ms}}{T_\mathrm{s}} J_{0\mathrm{s}\bm{k}} \tilde{\phi}_{\bm{k}}$. \textcolor{red}{NOTE: Although the Lenard-Bernstein model collision gkvp\_f0.48 operates on $\tilde{f}_{\mathrm{s}\bm{k}}$ but not on $\tilde{g}_{\mathrm{s}\bm{k}}$ due to historical reason, it will be modified near-future update.}

\underline{Lenard-Bernstein model collision operator}\\
\begin{align}
  C_{\mathrm{a}\bm{k}}^\mathrm{LB} = \nu_\mathrm{a} \left[ v_\mathrm{ta}^2 \frac{\partial^2 \tilde{g}_{\mathrm{a}\bm{k}}}{\partial v_\parallel^2} + v_\mathrm{ta}^2 \frac{\partial^2 \tilde{g}_{\mathrm{a}\bm{k}}}{\partial v_\perp^2} + v_\parallel \frac{\partial \tilde{g}_{\mathrm{a}\bm{k}}}{\partial v_\parallel} + \left(\frac{v_\mathrm{ta}^2}{v_\perp} + v_\perp \right) \frac{\partial \tilde{g}_{\mathrm{a}\bm{k}}}{\partial v_\perp} + 3 \tilde{g}_{\mathrm{a}\bm{k}} - k_\perp^2\rho_\mathrm{ta}^2 \tilde{g}_{\mathrm{a}\bm{k}} \right].
  \label{eq:LB_collision} 
\end{align}

\underline{Lorentz model collision operator}\\
\begin{align}
  C_{\mathrm{a}\bm{k}}^\mathrm{Lorentz} = \nu_\mathrm{D}^\mathrm{ab} &\Bigg[ \frac{v_\perp^2}{2} \frac{\partial^2 \tilde{g}_{\mathrm{a}\bm{k}}}{\partial v_\parallel^2} + \frac{v_\parallel^2}{2} \frac{\partial^2 \tilde{g}_{\mathrm{a}\bm{k}}}{\partial v_\perp^2} - v_\parallel v_\perp  \frac{\partial^2 \tilde{g}_{\mathrm{a}\bm{k}}}{\partial v_\parallel \partial v_\perp} - v_\parallel \frac{\partial \tilde{g}_{\mathrm{a}\bm{k}}}{\partial v_\parallel} + \frac{v_\perp}{2} \left(\frac{v_\parallel^2}{v_\perp^2} - 1 \right) \frac{\partial \tilde{g}_{\mathrm{a}\bm{k}}}{\partial v_\perp} \nonumber \\
&- \frac{k_\perp^2\rho_\mathrm{ta}^2}{4v_\mathrm{ta}^2} (2v_\parallel^2+v_\perp^2) \tilde{g}_{\mathrm{a}\bm{k}} \Bigg].
  \label{eq:Lorentz_collision}
\end{align}

\underline{Sugama model collision operator} \cite{Sugama2009PoP}\\
\begin{align}
  C_{\mathrm{a}\bm{k}}^\mathrm{Sugama} &= \sum_\mathrm{b} \left[ C^\mathrm{V}_\mathrm{ab}(\tilde{g}_{\mathrm{a}\bm{k}}) + C^\mathrm{D}_\mathrm{ab}(\tilde{g}_{\mathrm{a}\bm{k}}) + C^\mathrm{F}_\mathrm{ab}(\tilde{g}_{\mathrm{b}\bm{k}}) \right].
  \label{eq:Sugama_collision}
\end{align}
The test-particle differential term $C^\mathrm{V}_\mathrm{ab}$, the test-particle non-isothermal term $C^\mathrm{D}_\mathrm{ab}$, and the field-particle term $C^\mathrm{F}_\mathrm{ab}$ are given by,
\begin{align}
  C^\mathrm{V}_\mathrm{ab}(\tilde{g}_{\mathrm{a}\bm{k}}) &= \frac{\nu_\parallel^\mathrm{ab}v_\parallel^2+\nu_\mathrm{D}^\mathrm{ab}v_\perp^2}{2}\frac{\partial^2 \tilde{g}_{\mathrm{a}\bm{k}}}{\partial v_\parallel^2} + \frac{\nu_\mathrm{D}^\mathrm{ab}v_\parallel^2+\nu_\parallel^\mathrm{ab}v_\perp^2}{2}\frac{\partial^2 \tilde{g}_{\mathrm{a}\bm{k}}}{\partial v_\perp^2} + (\nu_\parallel^\mathrm{ab}-\nu_\mathrm{D}^\mathrm{ab})v_\parallel v_\perp \frac{\partial^2 \tilde{g}_{\mathrm{a}\bm{k}}}{\partial v_\parallel \partial v_\perp} \nonumber \\
  &+ \nu_\mathrm{g}^\mathrm{ab} v_\parallel \frac{\partial \tilde{g}_{\mathrm{a}\bm{k}}}{\partial v_\parallel} + \left[ \nu_\mathrm{g}^\mathrm{ab} + \frac{\nu_\mathrm{D}^\mathrm{ab}}{2} \left( 1 + \frac{v_\parallel^2}{v_\perp^2} \right) \right] v_\perp \frac{\partial \tilde{g}_{\mathrm{a}\bm{k}}}{\partial v_\perp} \nonumber \\
  &+\left[ \frac{\nu_\mathrm{h}^\mathrm{ab} x_\mathrm{a}^2}{2} - \frac{k_\perp^2}{4\Omega_\mathrm{a}^2} \left\{ \nu_\mathrm{D}^\mathrm{ab} (2v_\parallel^2 + v_\perp^2) + \nu_\parallel^\mathrm{ab} v_\perp^2 \right\} \right] \tilde{g}_{\mathrm{a}\bm{k}}, \\
  C^\mathrm{D}_\mathrm{ab}(\tilde{g}_{\mathrm{a}\bm{k}}) &= \sum_{j=1}^6 X_j^\mathrm{ab} M_j^\mathrm{ab}, \\ 
  C^\mathrm{F}_\mathrm{ab}(\tilde{g}_{\mathrm{b}\bm{k}}) &= \sum_{j=1}^6 Y_j^\mathrm{ab} M_j^\mathrm{ba},
\end{align}
where $x_\mathrm{a} = v / (\sqrt{2}v_\mathrm{ta})$, $\alpha_\mathrm{ab}=v_\mathrm{ta}/v_\mathrm{tb}$, $\nu_\mathrm{g}^\mathrm{ab} = \nu_\parallel^\mathrm{ab}x_\mathrm{a}^2(1-\alpha_\mathrm{ab})$, and $\nu_\mathrm{h}^\mathrm{ab} = 3\sqrt{\pi}\tau^{-1}_\mathrm{ab}\alpha_\mathrm{ab}\Phi'(x_\mathrm{b})/(4x_\mathrm{a}^2)$. The energy-diffusion and deflection frequencies are respectively given by $\nu_\parallel^\mathrm{ab} = 3\sqrt{\pi}\tau^{-1}_\mathrm{ab}G(x_\mathrm{b})/(2x_\mathrm{a}^3)$ and $\nu_\mathrm{D}^\mathrm{ab} = 3\sqrt{\pi}\tau^{-1}_\mathrm{ab}[\Phi(x_\mathrm{b})-G(x_\mathrm{b})]/(4x_\mathrm{a}^3)$ with the error function $\Phi(x) = \mathrm{erf}(x)$ and $G(x) = [\Phi(x) - x \Phi'(x)]/(2x^2)$. Expressions of the other coefficients $X_j^\mathrm{ab}$ and $Y_j^\mathrm{ab}$ and of the fluid moments $M_j^\mathrm{ab}$ are found, e.g., in the literature \cite{Nakata2015CPC}.




\section{Summary of formulation}
\label{sec:Summary of formulation}
Finally, one obtains the $\delta f$ gyrokinetic Vlasov-Poisson-Amp\`{e}re equations in a local flux-tube model, represented in perpendicular wave-number space,
\begin{align}
  &\frac{\partial \tilde{f}_{\mathrm{s}\bm{k}}}{\partial t} + \left( v_\parallel \nabla_\parallel + i \bm{k} \cdot \bm{v}_\mathrm{sG} + i \bm{k} \cdot \bm{v}_\mathrm{sC} \right) \left( \tilde{f}_{\mathrm{s}\bm{k}} + \frac{e_\mathrm{s} F_\mathrm{sM}}{T_\mathrm{s}} J_{0\mathrm{s}\bm{k}} \tilde{\phi}_{\bm{k}} \right) + N_{\mathrm{s}\bm{k}} \nonumber \\
  &- \frac{\mu \nabla_\parallel B}{m_\mathrm{s}} \frac{\partial}{\partial v_\parallel} \left( \tilde{f}_\mathrm{s} + \frac{e_\mathrm{s} F_\mathrm{sM}}{T_\mathrm{s}} J_{0\mathrm{s}} \tilde{\phi} \right) + \frac{e_\mathrm{s} F_\mathrm{sM}}{T_\mathrm{s}} \left[ v_\parallel \frac{\partial J_{0\mathrm{s}} \tilde{A}_\parallel}{\partial t} - i \bm{k} \cdot \bm{v}_{\mathrm{s}*} J_{0\mathrm{s}} (\tilde{\phi} - v_\parallel \tilde{A}_\parallel) \right] = C_{\mathrm{s}\bm{k}}, 
  \label{eq:vlasovink}\\
  &\left[ k_\perp^2 + \frac{1}{\varepsilon_0} \sum_\mathrm{s} \frac{e_\mathrm{s}^2 n_\mathrm{s}}{T_\mathrm{s}} \left( 1 - \Gamma_{0\mathrm{s}\bm{k}} \right) \right] \tilde{\phi}_{\bm{k}} = \frac{1}{\varepsilon_0} \sum_\mathrm{s} e_\mathrm{s} \int dv^3 J_{0\mathrm{s}\bm{k}} \tilde{f}_{\mathrm{s}\bm{k}},
  \label{eq:poissonink} \\
  &k_\perp^2 \tilde{A}_{\parallel\bm{k}} = \mu_0 \sum_\mathrm{s} e_\mathrm{s} \int dv^3 J_{0\mathrm{s}\bm{k}} v_\parallel \tilde{f}_{\mathrm{s}\bm{k}},
  \label{eq:ampereink}
\end{align}
where $J_{0\mathrm{s}\bm{k}}=J_0(k_\perp \rho_{\mathrm{s}})$ and $\Gamma_{0\mathrm{s}\bm{k}}=I_0(k_\perp^2 \rho_{\mathrm{ts}}^2)e^{-k_\perp^2 \rho_{\mathrm{ts}}^2}$ with 0th-order Bessel and modified Bessel functions $J_0$ and $J_1$. The included operators are again listed below,
\begin{align}
  \nabla_\parallel &= \frac{c_b}{B \sqrt{g}} \frac{\partial}{\partial z}, \\
  k_\perp^2 &= g^{xx} k_x^2 + 2 g^{xy} k_x k_y + g^{yy} k_y^2, \\
  i \bm{k} \cdot (\bm{v}_\mathrm{sG} + \bm{v}_\mathrm{sC}) &= \frac{m_\mathrm{s} v_\parallel^2 + \mu B}{e_\mathrm{s} c_b} \left( i K_x k_x + i K_y k_y \right) + i \frac{m_\mathrm{s} v_\parallel^2}{e_\mathrm{s} c_b} \frac{dP/dx}{B^2/\mu_0} k_y, \\
  i \bm{k} \cdot \bm{v}_\mathrm{s*} &= - i \frac{T_\mathrm{s}}{e_\mathrm{s} c_b} \left[ \frac{1}{L_{n\mathrm{s}}} + \left( \frac{m_\mathrm{s} v_\parallel^2}{2T_\mathrm{s}} + \frac{\mu B}{T_\mathrm{s}} - \frac{3}{2} \right) \frac{1}{L_{T\mathrm{s}}} \right] k_y, \\
  \mathcal{N}_{\mathrm{s}\bm{k}} &= - \sum_{\bm{k}'} \sum_{\bm{k}''} \delta_{\bm{k}'+\bm{k}'',\bm{k}}\frac{\bm{b} \cdot \bm{k}' \times \bm{k}''}{c_b} J_{0\mathrm{s}\bm{k}'} \left( \tilde{\phi}_{\bm{k}'} - v_\parallel \tilde{A}_{\parallel\bm{k}'} \right) \left( \tilde{f}_{\mathrm{s}\bm{k}''} + \frac{e_\mathrm{s} F_\mathrm{Ms}}{T_\mathrm{s}} J_{0\mathrm{s}\bm{k}''} \tilde{\phi}_{\bm{k}''} \right).
\end{align}
The coefficients for magnetic drift $K_x, K_y$ are given by Eqs. (\ref{eq:kkx}) and (\ref{eq:kky}), and the collision operator $C_{\mathrm{s}\bm{k}}$ is given by one of Eqs. (\ref{eq:LB_collision}) - (\ref{eq:Sugama_collision}).

\begin{thebibliography}{99}
\bibitem{Watanabe2006NF}
  T.-H. Watanabe, and H. Sugama,
  Nucl. Fusion {\bf 46}, 24 (2006).
\bibitem{Garbet2010NF}
  X. Garbet, Y. Idomura, L. Villard, and T.-H. Watanabe,
  Nucl. Fusion {\bf 50}, 043002 (2010).
\bibitem{Beer1995PoP}
  M. A. Beer, S. C. Cowley, and G. W. Hammett,
  Phys. Plasmas {\bf 2}, 2687 (1995).
\bibitem{Sugama2009PoP}
  H. Sugama, T.-H. Watanabe, and M. Nunami,
  Phys. Plasmas {\bf 16}, 112503 (2009).
\bibitem{Nakata2015CPC}
  M. Nakata, M. Nunami, T.-H. Watanabe, and H. Sugama,
  Comput. Phys. Commun. {\bf 197}, 61 (2015).
\end{thebibliography}